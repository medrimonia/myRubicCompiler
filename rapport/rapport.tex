\documentclass[12pt]{article}

\usepackage{geometry}
\usepackage[T1]{fontenc}
\usepackage[utf8]{inputenc}
\usepackage[francais]{babel}

\usepackage{graphicx}
\usepackage{lmodern}
\usepackage{color}
\usepackage{listings}
\lstset{language=SQL, frame=shadowbox, rulesepcolor=\color{white}}

\geometry{margin=2cm}


\begin{document}

\thispagestyle{empty}
%\noindent\includegraphics[width=0.25\textwidth]{enseirb-matmeca}

\vspace{\stretch{1}}

\begin{center}
	\Huge{\textbf{Rapport de projet Compilation :}}
\end{center}

\vspace{\stretch{2}}

\begin{tabular}{r@{:~}l}
	\textbf{Auteur} & \textit{HOFER Ludovic}\\
 \textbf{Encadrant} & \textit{BARTHOU Denis}\\
\end{tabular}

\vspace{\stretch{1}}

\begin{center}Deuxième année, filière informatique

	Date : \today
\end{center}

\newpage

\section{Introduction}

\section{Organisation du travail}

\subsection{Développement brique par brique}

\subsection{Tests}

\section{Les conteneurs}

\subsection{Hashmap}

\subsection{Dictionnary}

\subsection{LinkedList}

\subsection{La structure arborescente}

\section{Les aspects de la compilation}

\subsection{Les allocations mémoires}

\subsection{Les fonctions}

\subsection{Les opérateurs}

\subsection{Première gestion des types}

\subsubsection{Des listes de types autorisés}

\subsubsection{Contraintes lors du parcours de l'arbre}

\subsubsection{Choix du type lors de la génération de code}

\subsection{Les conditionnelles}

\subsection{Les boucles}

\subsubsection{La boucle \em{while}}

\subsubsection{La boucle \em{for}}

\subsection{Gestion de la surcharge}

\subsection{La gestion des classes}

\section{Conclusion}


\end{document}
