\documentclass[12pt]{article}

\usepackage{geometry}
\usepackage[T1]{fontenc}
\usepackage[utf8]{inputenc}
\usepackage[francais]{babel}

\usepackage{graphicx}
\usepackage{lmodern}
\usepackage{color}
\usepackage{listings}
\lstset{language=SQL, frame=shadowbox, rulesepcolor=\color{white}}

\geometry{margin=2cm}


\begin{document}

\thispagestyle{empty}
%\noindent\includegraphics[width=0.25\textwidth]{enseirb-matmeca}

\vspace{\stretch{1}}

\begin{center}
	\Huge{\textbf{Rapport de projet de Compilation :}}
\end{center}

\vspace{\stretch{2}}

\begin{tabular}{r@{:~}l}
	\textbf{Auteur} & \textit{HOFER Ludovic}\\
  \textbf{Encadrant} & \textit{BARTHOU Denis}\\
\end{tabular}

\vspace{\stretch{1}}

\begin{center}Deuxième année, filière informatique

	Date : \today
\end{center}

\newpage

\section{Introduction}
Ce travail s'inscrit dans le cadre du cours de compilation dispensé à
l'enseirb-matmeca. Le but qui nous a été donné est de réaliser un compilateur
capable de transformer un programme écrit en rubic\footnote{Une version
simplifiée du langage Ruby} en une version du programme respectant les
spécifications du langage llvm\footnote{http://llvm.org/}.\\

Afin de nous faciliter le travail, l'analyse lexicale et syntaxique étaient
déjà disponibles sous la forme d'un fichier lex et d'un fichier yacc. Le
travail à fournir était donc plus axé sur la génération de code.\\

Outre le travail classique à fournir lors de l'implémentation d'un
compilateur pour un langage simplifié, plusieurs difficultés techniques
étaient présentes dans ce langage :

\begin{itemize}
\item Résolution du types des variables\footnote{Le type de celles-ci
  n'étant pas fourni, il est nécessaire de le déterminer à la compilation}
\item Gestion de la surcharge\footnote{Il est possible d'écrire plusieurs
  fonctions ayant le même nom, si elles diffèrent par leurs paramètres}
\item Gestion des classes\footnote{rubic étant un langage objet,
  il doit être possible de déclarer des classes et de les utiliser}
\item Génération de fonction ambigues\footnote{Lorsque les types de certains
  paramètres d'une fonction ne peuvent pas être résolus directement, il faut
  générer le code de plusieurs fonctions pour une seule définition}
\end{itemize}

Certaines de ces tâches étant d'une complexité non négligeable, il est
intéressant de noter comment elles ont été traitées ainsi que les raisons
de certains choix d'implémentation.\\

Au vu de l'ampleur de la tâche\footnote{Bien que ce soit le projet du premier
semestre sur lequel j'aie le plus travaillé, PFA excepté, je suis loin de
remplir de façon satisfaisante les objectifs qui ont été donnés.}, il m'a
semblé évident qu'un minimum d'organisation était nécessaire et que le choix
des méthodes de développement utilisées était primordial. La méthode utilisée
sera donc détaillée dans ce document.\\

Ayant déjà fait un travail\footnote{http://code.google.com/p/my-c-compiler/}
un peu similaire lors de ma scolarité à l'université de Bordeaux 1
\footnote{Le projet mentionné ici ne comportait pas les mêmes difficultés,
puisque le langage d'entrée était typé et ne permettait pas la surcharge.},
j'ai décidé de rester en c plutôt que de choisir le java\footnote{C'est le
choix que j'avais pour mon projet précédent, y utilisant jflex et cup.}. Les
principales difficultés engendrées par ce changement étaient les suivantes :

\begin{itemize}
\item L'utilisation de type abstrait de données\footnote{En java ceux-ci sont
déjà fournis, alors qu'en c, il est plus difficile de trouver la version
adaptée, j'ai donc écrit mes conteneurs moi-même.}
\item La gestion de la mémoire\footnote{Celle-ci étant de loin plus facile en
java.}
\end{itemize}

\section{Organisation du travail}

Sachant pertinemment que j'avais peu de chance d'arriver à remplir toutes les
exigences du projet en étant seul dans mon groupe et qu'il m'était difficile
d'appréhender directement tout le contenu du projet, j'ai préféré m'orienter
vers une approche de construction progressive du projet en ajoutant une à une
des fonctionnalité plutôt que de concevoir le projet en entier puis de passer
à une phase d'implémentation et de me retrouver coincé par des problèmes que
je n'avais pas envisagé.

\subsection{Développement brique par brique}
Le principe de l'ordre de développement choisi est assez simple, il consiste
à se fixer comme objectif la compilation d'un certain programme rubic
\footnote{En commençant évidemment par les plus simples.}, puis d'ajouter le
code manquant. Ainsi, le développement alterne à un rythme assez élevé les
phases de conception, de codage et de débuguage.\\

Cette méthode présente les avantages suivants :
\begin{itemize}
\item Une stimulation par une visualisation de progression aisée.
\item L'assurance qu'à la fin du projet, la conception sera à peu près au
  même niveau que l'implémentation.
\item La conception d'une notion complexe se basera sur l'implémentation de
  notions plus simples au lieu de se baser sur la conception de celle-ci.
\end{itemize}

Cependant, elle possède aussi des désavantages :
\begin{itemize}
\item L'ajout de nouvelles fonctionnalités peut obliger à retravailler de
  manière plus ou moins marquée le code existant.
\item En implémentant de nouvelles fonctionnalités, il est possible de
  dégrader celles déjà présentes sans s'en apercevoir.
\end{itemize}

Afin de pallier le premier problème, j'ai essayé de garder à l'esprit les
fonctionnalités complexes lorsque j'implémentais celles qui étaient plus
simple.\\

Pour éviter d'introduire des bugs lors du développement, j'ai fait évoluer
en parallèle un module de tests me permettant d'assurer la transformation de
code rubic en code exécutable.

\subsection{Tests}

Ma première approche des tests a été de chercher à maitriser la
transformation de code llvm en exécutable, cette partie n'étant pas
directement reliée au projet, il n'était pas nécessaire d'écrire le moindre
code pour celle-ci. Il suffisait de prendre en main les outils de compilation
existants, j'en ai profité pour ajouter les instructions génériques
nécessaires dans le Makefile afin de transformer facilement n'importe quel
fichier rubic en exécutable.\\

Une fois que le projet avait un peu plus pris forme, j'ai écrit un script
permettant de compiler tous les fichiers {\em.rubic}, de les lancer et de
vérifier que leur valeur de retour est bien égale à zéro. Ceci permet de bien
s'assurer de ne pas avoir introduit de bug.\\

Dans un second temps, un affichage du nombre de tests total et du nombre de
tests ayant compilé a été ajouté afin de déceler plus facilement si des
problèmes surviennent à la compilation des fichiers rubics.

\section{Les conteneurs}

Étant donné qu'une grande partie des données que le compilateur doit stocker
ne sont pas de taille fixée, il semble évident qu'utiliser uniquement des
tableaux ne sera pas adapté. Les conteneurs génériques étant difficiles à
trouver en c, j'ai donc implémenté les miens en essayant de les rendre
réutilisable afin de pouvoir m'en servir à nouveau pour d'autres projets.

\subsection{Hashmap}
Ce choix s'est imposé comme étant indispensable pour la gestion du stockage
de différentes données telles que les fonctions ou les variables. Il est
beaucoup plus aisé d'utiliser une structure permettant un accès et une
insertion rapide.\\

Dans une première approche, je souhaitais effacer automatiquement les clés et
les objets contenus, mais je me suis rapidement rendu compte qu'il valait
souvent mieux laisser l'utilisateur du conteneur faire ses free.\\

J'ai aussi du ajouter un moyen d'itérer sur ces conteneurs afin d'allouer
toutes les variables par exemple.\\

La possibilité d'ajuster la taille d'une hashmap a été laissée de côté
pour l'instant, les problèmes de temps d'accès n'étant pas cruciaux dans le
cadre actuel.

\subsection{Dictionnary}
Comme de nombreuses Hashmap auraient prit comme clé des chaînes de caractère,
j'ai préféré créer un conteneur qui simplifie les pointeur renvoyés et
qui fournit dans le même temps une fonction de hachage et une fonction
vérifiant l'égalité de deux chaînes de caractère tout en respectant le
prototype souhaité par les Hashmap.

\subsection{LinkedList}
Afin de simuler différentes données n'ayant pas de clé, mais dont le nombre
reste tout de même inconnu\footnote{La possibilité de créer de nouveau types
étant offerte en rubic via les classes, le nombre de types possibles est
inconnu par exemple.}, j'ai décidé de choisir des listes chaînées, celles-ci
étant relativement simple à implémenter et agréable d'utilisation. De plus,
il m'était nécessaire d'avoir une structure semblable, afin de garantir que
les insertions dans la hashmap fonctionnent.

\subsection{La structure arborescente}
Étant donné que l'on parle d'arbre syntaxique, il est assez naturel que j'ai
utilisé des arbres pour représenter le résultat de l'analyse syntaxique. Fait
pratique, comme tout noeud représente un arbre et que je n'avais pas besoin
de retirer des noeuds de l'arbre, il ne m'était pas nécessaire de fournir
de nombreuse primitives pour le maniement de cette structure. Au contraire,
je l'ai gardée assez explicite avec de nombreux champs spécifique faisant
partie des noeuds, cela m'a permit une écriture simplifiée dans certains cas.

\subsection{DoublyLinkedList}
Estimant nécessaire d'avoir à disposition une liste pouvant être parcourue
dans les deux sens, j'ai ajouté à ces conteneurs une implémentation de liste
doublement chaînée utilisant le principe de sentinelles.\\

Cependant, avec le recul, je me demande s'il était réellement nécessaire de
disposer de ces fonctionnalités ou si une liste chaînée n'aurait pas suffit
pour gérer le polymorphisme.

\section{Les aspects de la compilation}
Compiler un programme rubic en programme llvm pouvant être perçu comme un
ensemble de fonctionnalités à implémenter et donc comme un ensemble de
problèmes. Dans cette optique, je vais expliquer les méthodes utilisées pour
chacun des sous-problèmes.\\

Il est à noter qu'afin de simplifier un maximum les problèmes, j'ai commencé
par considérer que le seul type accepté était les int, ainsi toutes mes
opérations pouvaient être codée plus facilement.

\subsection{Les allocations mémoires}
Afin d'allouer de la mémoire à toutes les différentes variables qui en
nécessitaient, j'ai décidé de simpler allouer l'espace à toutes les
variables au début de chaque fonction, cependant, je n'ai pas vérifié si
ces espaces mémoires étaient bien libérés à la sortie de chaque fonction.

\subsection{Les fonctions}
Pour traiter les fonctions, il était nécessaire de distinguer leurs
paramètres des simples variables locales, les premiers appartennant à la
signature tandis que les secondes n'ont qu'une visibilité locale.\\

Traité d'une manière similaire aux variables, les fonctions étaient
simplement ajoutées à un dictionnaire et pour déterminer si l'appel était
valide, il fallait simplement vérifier si la fonction était présente dans le
dictionnaire et si les paramètres correspondaient à ce qui était attendu.

\subsubsection{Les built-in}
L'utilisation de certains {\em built-in} comme {\em puts} nécessitait que les
fonctions soient déclarées, même si elles n'étaient pas définies. Afin que
cela reste le plus simple pour l'utilisateur, j'ai simplement ajouté un
dictionnaire contenant les {\em built-in}, ainsi qu'un autre contenant les
{\em built-in} utilisé\footnote{Celui-ci étant évidemment initialisé comme
vide}. Ainsi à chaque fois qu'une utilisation de {\em built-in} était
détectée, il était assuré que le built-in serait déclaré dans le fichier
généré. Cette procédure m'a permit de garantir que si de nouveaux
{\em built-in} étaient ajouté, il n'y aurait pas un nombre de lignes inutiles
grandissant au début de chaque fichier généré.

\subsection{Les opérateurs}


\subsection{Première gestion des types}
Afin de travailler dans un environnement simplifié, j'ai commencé par
considérer comme types possibles uniquement les int et les chaînes de
caractère, l'ambiguité entre ces deux types étant peu probable. Ainsi,
j'ajoutai malgré tout la prise en charge de plusieurs types, sans pour autant
m'heurter à des problèmes plus complexes\footnote{L'ajout de float engendrait
en effet une obligation de traiter (de manière plus ou moins complète) le cas
du polymorphisme}.

\subsubsection{Des listes de types autorisés}
Afin de déterminer de quel type est une variable, j'ai pensé qu'il était plus
simple de commencer par affirmer qu'elle pouvait être de tous les types
connus, puis de restreindre ces types lorsqu'une contrainte était découverte
\footnote{Que ce soit sous la forme d'un opérateur, d'un appel à une fonction
ou n'importe quel autre type de contrainte}. Lorsqu'un nouveau noeud était
créé, les types qui lui étaient autorisé dépendait donc du type de ses fils.
Le nouveau noeud exerçait aussi une contrainte sur les noeuds qui étaient
directement ses fils, mais pas sur toute sa descendance\footnote{Ceci afin
d'éviter de faire exploser la complexité. La compilation étant sensé être une
opération assez rapide.}.

\subsubsection{Contraintes lors du parcours de l'arbre}
En parcourant l'arbre plusieurs fois, il devenait donc possible de répéter
les contraintes plusieurs fois et d'ainsi affiner le résultat dans des cas
particuliers un peu litigieux. Cependant, étant donné qu'il est difficile
de borner le nombre de passages nécessaires, un seul passage était effectué
au début\footnote{Les changements seront plus détaillés lors de la phase
gestion du polymorphisme}. Le choix pris à ce moment était de déclencher une
erreur de compilation lorsque le type des paramètres d'une fonction n'étaient
pas déterminés après le premier passage.

\subsubsection{Choix du type lors de la génération de code}
Lors de la passe sur l'arbre servant à effectuer la génération de code,
il m'était nécessaire de décider quel type allait être utilisé pour le noeud,
étant donné que celui-ci déterminait le nom de l'opération à appliquer
\footnote{add, fadd ou icmp, fcmp etc...}. Pour ceci, j'ai décidé d'utiliser
une logique extrêment simple, s'il restait un seul type alloué, c'est celui
qui était utilisé et dans le cas contraire, il y avait une erreur de
compilation\footnote{Le type de la variable étant ambigu ou impossible
à définir}.

\subsection{Les conditionnelles}
Afin d'implémenter les structures de code conditionnelles\footnote{blocs if
et blocs if/else}, il a été évidemment nécessaire de supporter les booléens,
ce qui s'est fait assez facilement, puisqu'avec la manière dont j'avais
implémenté mon gestionnaire de type, cela ne représentait que quelques lignes
à ajouter.\\

Afin de simplifier un peu l'implémentation, j'ai décidé que les deux formes
de blocs conditionnels seraient représentés par le même noeud. Le contenu de
celui-ci serait simplement représenter par trois noeuds, la condition, la
liste d'instructions à exécuter dans le cas vrai et celle dans le cas faux
\footnote{Cette dernière sera donc mise à nulle dans les cas où le elsen'est
pas utilisé}\\

La présence des conditionnelles a aussi grandement facilité la réalisation
des tests de non régression, puisque cela permettait de renvoyer zéro
uniquement dans le cas que l'on valide \footnote{il n'était donc plus
nécessaire que le résultat soit zéro. Ceci est extrêmement pratique, pour les
tests sur les floats par exemple.}\\

Afin que les conditionnels fonctionnent il a aussi été nécessaire de garantir
que les labels générés soient différent. Mon choix pour a simplement été
de conserver une variable globale indiquant le numéro de label et de
l'incrémenter à chaque utilisation de label.

\subsection{Les boucles}
La gestion des boucles s'est révélée relativement simple à implémenter,
le principal étant d'assurer que les sauts soient dirigés vers les bons
labels et que les labels soient bien différents.

\subsubsection{La boucle \em{while}}
Une boucle while consiste simplement à générer un label L1 avant la
génération du code contenu dans la condition, un label L2 au début du code
interne au while et un label L3 après celui-ci. Il faut aussi la présence
d'un saut conditionnel juste avant L2 qui ira sur L2 dans le cas vrai et L3
dans le cas où la condition se révèle fausse. À la fin du code interne au
while il y aura aussi la présence d'un saut à destination de L1, afin que la
condition soit évaluée à nouveau.

\subsubsection{La boucle \em{for}}
Cette structure de contrôle ressemble énormément au while, la différence
principale étant que le code de condition n'est pas écrit de la même manière.
Cependant, j'ai géré une fonctionnalité supplémentaire pour les boucles for
en prenant en compte que les variables peuvent être internes au bloc et donc
être d'un autre type à l'extérieur du bloc si elles n'ont pas été déclarées
avant\footnote{Du moins, c'était mon but, mais cette partie n'est pas
encore finalisée et le test qui y est associé ne compile pas}.

\subsection{Gestion du polymorphisme}
Lorsque l'on ajoute la simple présence du type float à la liste des types
existants, le problème du polymorphisme se pose déjà sur des problèmes très
peu complexe.
% code equal
Dans cette fonction, il y a déjà une ambiguité sur les types à utilisé pour
les paramètres, en effet, a et b peuvent être des ints comme des floats. On
peut aussi voir que les combinaisons int,float et float,int ne sont pas
valides.

\subsubsection{Itération sur toutes les possibilités}
Il y a plusieurs façon de gérer le cas des déclaration de fonctions avec des
paramètres dont le type est ambigu. On peut choisir de générer les fonctions
uniquement lorsqu'elles seront appelées, ainsi le code correspondant à un
prototype ne sera généré que si la fonction est utilisée. Cependant, ce qui
m'a dissuadé de choisir cette méthode est le fait que si le code généré a
vocation à être utilisé par un autre module, la fonction désirée ne sera pas
forcément accessible. J'ai donc préféré itérer sur toutes les combinaisons
possibles et générer celles qui engendrent un code valide.\\

Il reste important de souligner qu'itérer ainsi sur toutes les possibilités
engendrent une complexité exponentielle en fonction du nombre de paramètres
\footnote{Ceci dans le pire des cas, même s'il n'est pas sensé arriver
souvent, cela engendre le fait qu'il est facile d'écrire un programme dont
la compilation prendra un temps incroyablement long.}

\subsubsection{Génération du code}
Afin de générer le code adapté, il est nécessaire d'appliquer les
modifications de type des variables avant de générer le code. La méthode
que j'ai utilisée consiste à enregistrer les prototypes valides durant une
première phase, puis lors de la génération du code, le choix du type est
appliqué aux paramètres, le type des noeuds est mis à jour et, ensuite
seulement, le code est écrit. Cette méthode permet de ne pas avoir à recopier
tout l'arbre d'une fonction uniquement pour changer le prototype utilisé.\\

J'ai décidé pour cette partie de générer une erreur de compilation uniquement
si aucun prototype valide n'était trouvé. Dans les cas où certains prototypes
ne fonctionneraient pas, ils ne généreraient simplement pas de code.\\

Cette phase ayant été particulièrement ardue à débuguer, je me suis servi de
quelques printf écrivant des commentaires en langage llvm pour simplifier
un premier repérage des erreurs, mais mon outil principal a été gdb qui m'a
permit de rapidement détecté mes erreurs de codage.

\subsubsection{La surcharge en llvm}
Comme le langage llvm ne supporte pas la surcharge, il a été nécessaire de
créer une différence de noms entre les différentes fonctions qui partageaient
le même noms. J'ai simplement choisi de préfixer tous les noms de fonctions
par le nom des types des arguments.

\subsection{La gestion des {\em floats}}
Lors de l'étape précédente, j'ai pu ajouter les {\em floats} en tant que
type, mais cela ne suffisait pas à leur gestion, effectivement, le
compilateur llvm utilisé butait sur l'écriture des éléments et les tests ne
compilaient donc pas.\\

Après quelques recherches, j'ai choisi d'utiliser la représentation
hexadécimal des {\em floats} et grâce à cela j'ai réussi à faire compiler le
programme de tests, cependant, mon test ne donnait pas le résultat escompté.
\\

Un autre groupe m'a soufflé à ce moment là que le problème disparaissait
purement et simplement si l'on se servait des {\em double} à la place des
{\em floats}, j'ai donc essayé et approuvé cette solution.

\subsection{La gestion des classes}
N'ayant malheureusement pas réussi à trouver le temps d'implémenter cette
partie, je n'ai pas grand chose à dire sur celle-ci.

\section{La gestion de la mémoire}
Un des aspects critique présenté lors de la gestion de ce programme en c par
rapport à mon expérience précédente en java est évidemment la gestion de
la mémoire. Or, si j'ai réussi à éviter la présence de fuites mémoires lors
des premiers tests qui n'utilisaient que des fonctionnalités assez simple,
tout est devenu bien plus complexe lors de l'implémentation des parties plus
difficile, et ce à tel point que j'ai choisi de repousser le moment de la
gestion mémoire afin de réussir à implémenter plus rapidement ce que je
désirais. En effet, obtenir une solution présentant des fuites mémoires était
déjà quelque chose de satisfaisant pour moi et je craignais qu'à vouloir
viser le mieux directement, je n'arrive même pas au bien.

\subsection{Premier nettoyage des fuites mémoires}
Travail de longue haleine, le nettoyage des fuites mémoires m'a demandé un
temps non négligeable, mais malgré tout, j'ai mis moins de temps que ce que
j'avais estimé dans un premier temps. Pour cette tâche, j'ai choisi d'utiliser
la procédure suivante.\\

J'ai commencé par choisir un fichier de test qui allait me servir de
référence. Je l'ai compilé avec un valgrind simple afin de voir rapidement la
proportion de free par rapport au nombre d'allocs. Comme je m'y attendais,
celle-ci était particulièrement faible. Ensuite, j'ai obtenu plus de détails
à l'aide d'un {\em valgrind --leak-check=full --show-reachable=yes}. Ces
détails me permettaient de savoir quelle partie de la mémoire n'était pas
libérée et comme je connaissais l'intégralité de mon code, je pouvais assez
facilement savoir quand et comment la libérer\footnote{Il est à noter que
j'ai tout de même pu prendre conscience que ce travail aurait réellement été
affreux si j'avais du le faire sur du code qui m'était inconnu et je pense
donc qu'il est nettement préférable lorsque l'on travaille en équipe que
chacun veille constamment à ne pas laisser trop de fuites mémoires.}.\\

Même si ainsi présentée, la situation peut sembler un peu idyllique, je me
dois d'avouer que certains problème de fuites mémoires ont tout de même
nécessité un peu plus de raffinement qu'un simple free bien placé. À titre
d'exemple, je citerai la gestion des identifiants. Ceux-ci étant présent à
différents endroits\footnote{En tant que clé de dictionnaire, de contenu de
certains noeuds, de nom de fonction etc...}, j'ai choisi de simplement
maintenir à jour une liste contenant tous ces identifiants, et d'ainsi tous
les libérer de manière uniforme à la fin de l'exécution.\\

Cette partie du travail m'a aussi encouragé à uniformisé le traitement en
mémoire, en effet, pour les built-ins ou pour les fonctions décrites par
l'utilisateur, le fonctionnement n'était pas le même au niveau de la mémoire
avant que je nettoie un peu le code.

\section{Conclusion}
Ce projet a sans conteste été l'occasion d'apprendre beaucoup et de me
heurter à plusieurs problématiques difficiles, comme je m'y attendais, le
volume de travail était conséquent et ma plus grande déception est que je
sais que mon projet ne présente pas une gestion particulièrement propre de
certains problèmes. Cependant, cette première approche un peu brutal était
nécessaire pour prendre conscience de certains problèmes qui ne me sont
apparus que lors de l'implémentation.\\

Au terme du temps accordé à ce projet, je pense que j'éprouverais du plaisir
à travailler sur la réalisation d'un compilateur en équipe
\footnote{Effectivement, pour certaines problématiques, le fait d'avoir un
dialogue et l'avis de plusieurs personnes était sans aucun doute d'une
utilité indiscutable} et avec un temps disponible pour ce travail plus
important que celui que j'avais à ma disposition. Pour arriver à un résultat
propre, donnant des informations précises lors de tous les types d'erreurs
de compilation, je pense qu'il aurait fallu plusieurs fois le temps que j'ai
accordé à ce projet.\\

Je suis particulièrement content que malgré le fait que j'ai travaillé en
monôme, j'ai réussi à échanger avec d'autres groupes, leur apportant un peu
de mes idées et piochant certaines des leurs. Je pense effectivement qu'il
est indispensable de savoir reconnaître quand les autres ont de bonnes idées
en informatique si l'on veut progresser et réaliser du travail de qualité.


\end{document}
